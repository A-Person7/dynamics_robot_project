\documentclass[nofoot,pdf-a,balance,upint,subscriptcorrection,varvw]{asmeconf}
\special{papersize=8.5in,11in}

\usepackage{amsmath}
\usepackage{mathtools}
\usepackage{amssymb}
\usepackage{xfrac}
\usepackage[margin=1.00in]{geometry}
\usepackage{tabto}
\usepackage{tikz}


\begin{document}

    \ConfName{Proceedings of the Dynamics 2024 Very Local Mechanical Engineering Congress and Exposition}
    \ConfAcronym{ENSC-306-01 Spring 2024}
    \ConfDate{Spring, 2024}
    \ConfCity{Spokane, Wa}
    
    \title{Robot Arm Project}
    \SetAuthors{Lucas Johnston\JointFirstAuthor, Brendan Moskalik\JointFirstAuthor}
    \date{\today}

    \maketitle

    \section*{Problem Description}
	
    \tab Given an egg in free fall, we want to determine a smooth path over time that a robotic arm can follow to catch the egg without it breaking. The robotic arm consists of a linear actuator and a motor that rotates the linear actuator such that it's motion can be fully described by $r\left(t\right)$ and $\theta\left(t\right)$, which are the radius as a function of time and the angle as a function of time, respectively. In the name of good faith and best practice, all angles discussed will be in radians. Together with the physical constraints imposed on the system that lock the robotic arm's catcher onto a straight line, these both fully define a function $y\left(t\right)$, which describes the height of the catcher. Helpfully, $y\left(t\right)$ fully describes both $r\left(t\right)$ and $\theta\left(t\right)$, so our primary goal, first and foremost, is to solve for a valid $y\left(t\right)$ function. The arm must start at rest 0.6 meters above the table surface. To smoothly catch the egg, the arm must match the position, velocity, and acceleration of the egg at some instance around 0.5 meters above the table. The egg starts at rest 0.8 meters above the table.\newline \newline 
    In other words, $y:\left[0, \infty\right) \to \left(-\infty, 0\right)
$ must satisfy the following conditions:
    \begin{equation}
        y\left(0\right) = 0.8
    \end{equation}
    \begin{equation}
        \dot{y}\left(0\right) = 0
    \end{equation}
	
		\section*{Assumptions}
	
	\begin{itemize}
		\item We are on earth and that $g = 9.81 \frac{m}{s^2}$
		\item There is negligable drag
		\item There is negligable flex in the robot arm
	\end{itemize}

	\section*{Solution Methods}
	
	\section*{Solution}
	
	\section*{Solution Validation}
	
	\section*{Discussion of Results}
	
	\tab Our solution meet all the requirements given by the problem statement. We only broke the motion of $y\left(t\right)$ into two segments for the egg. There is the motion before contact with the robot where it is accelerating down, and after when it starts to slow down. The robot arm itself traveling on this path was made to be one segment only as we extrapolated the motion of slowing down the egg to before contact with it. This solution is a very smooth way to do it, as we ended up dealing with a constant snap function. Because it is fundamentally a constant function over the path it also is a very simple solution to the problem.
	% needs data of actuators
	
\end{document}
